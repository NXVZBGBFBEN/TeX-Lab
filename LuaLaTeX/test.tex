\documentclass{ltjsarticle}
\usepackage{amsmath}
\usepackage{tikz}
\usetikzlibrary{patterns}

%数値定義
\newcommand{\radius}{1cm}
\newcommand{\ratio}{0.8}

%円の書式定義
\newcommand{\circleA}{(90:\ratio*\radius) circle [radius=\radius]}
\newcommand{\circleB}{(-150:\ratio*\radius) circle [radius=\radius]}
\newcommand{\circleC}{(-30:\ratio*\radius) circle [radius=\radius]}

%円のラベル定義
\tikzset{set label/.style={fill=white,circle,inner sep=.5mm}}
\newcommand{\drawLabels}{
  \node[set label] at (90:\radius+\ratio*\radius) {A};
  \node[set label] at (-150:\radius+\ratio*\radius) {B};
  \node[set label] at (-30:\radius+\ratio*\radius) {C};
}

%ベン図全体の描画定義
\newcommand{\drawVenn}[2]{
  \begin{tikzpicture}[baseline=0pt]
    \path[use as bounding box] (0,0) circle [radius=2.5*\radius];
    \begin{scope}[color=gray]
      #1
    \end{scope}
    \begin{scope}[color=white]
      #2
    \end{scope}
    \draw \circleA \circleB \circleC;
    \drawLabels
  \end{tikzpicture}
}

\begin{document}
  \begin{center}
    \begin{align*}
      &
      \drawVenn{\clip \circleC; \fill \circleA; \fill \circleB;}{}
      \drawVenn{\clip \circleC; \fill \circleA;}{}
      \drawVenn{\clip \circleB; \fill \circleC;}{}\\
      &
      \drawVenn{\clip \circleC \circleA; \fill \circleB; \fill \circleC;}{}
      \drawVenn{\fill \circleC; \fill \circleA;}{}
      \drawVenn{\fill \circleB; \fill \circleC;}{}\\
      &
      \drawVenn{\fill[even odd rule] \circleA \circleB \circleC;}{\clip \circleA; \clip \circleB; \clip \circleC; \fill \circleA;}
    \end{align*}
  \end{center}
\end{document}